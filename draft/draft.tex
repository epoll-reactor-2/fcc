\documentclass{article}

\usepackage{amsmath}
\usepackage{syntax}
\usepackage{fancyhdr}
\usepackage{array}

\newcolumntype{L}[1]{>{\raggedright\let\newline\\\arraybackslash\hspace{0pt}}m{#1}}
\newcolumntype{C}[1]{>{\centering\let\newline\\\arraybackslash\hspace{0pt}}m{#1}}
\newcolumntype{R}[1]{>{\raggedleft\let\newline\\\arraybackslash\hspace{0pt}}m{#1}}

\title{Weak language draft}
\author{epoll-reactor}
\date{since December 2021}

\begin{document}
	\pagestyle{fancy}
	\rhead{\leftmark}
	\lhead{Weak language}
	
	\maketitle
	\tableofcontents
	
	\newpage
	
	\section{Scope}
		This document describes requirements for implementation of weak
		programming language.
		
	\section{Lexical elements}
		\upshape
		
		\subsection{Keywords}
			\begin{tabular}{ R{3cm} R{3cm} R{3cm} }
				\textbf{boolean}  & \textbf{break}  & \textbf{char}   \\
				\textbf{continue} & \textbf{do }    & \textbf{false}  \\
				\textbf{float}    & \textbf{for}    & \textbf{if}     \\
				\textbf{int}      & \textbf{return} & \textbf{string} \\
				\textbf{true}     & \textbf{void}   & \textbf{while}  \\
			\end{tabular}
		
		\subsection{Operators and punctuators}
			\begin{tabular}{ R{1.5cm} R{1.5cm} R{1.5cm} R{1.5cm} R{1.5cm} R{1.5cm} R{1.5cm} }
				\tt{=}   & \tt{*=}     & \tt{/=}  & \tt{\%=} & \tt{+=}  & \tt{-=}   \\
				\tt{<<=} & \tt{>>=}    & \tt{\&=} & \tt{|=}  & \tt{\^=} & \tt{\&\&} \\
				\tt{|}   & \textbf{\^} & \tt{\&}  & \tt{==}  & \tt{!=}  & \tt{>}    \\
				\tt{<}   & \tt{>=}     & \tt{<=}  & \tt{<<}  & \tt{>>}  & \tt{+}    \\
				\tt{-}   & \tt{*}      & \tt{/}   & \tt{\%}  & \tt{++}  & \tt{--}   \\
				\tt{[}   & \tt{]}      & \tt{(}   & \tt{)}   & \tt{\{}  & \tt{\}}   \\
			\end{tabular}
		
	\section{Grammar summary}
		\itshape
		\setlength{\grammarindent}{12em}

		\begin{grammar}
			<program> ::= <function-declaration>*
			
			<function-declaration> ::= <ret-type> <id> \textbf{(} <parameter-list-opt> \textbf{)} \textbf{\{} <stmt>* \textbf{\}}

			<ret-type> ::= <type>
			\alt <void-type>
			
			<type> ::= \textbf{int}
			\alt \textbf{float}
			\alt \textbf{char}
			\alt \textbf{string}
			\alt \textbf{boolean}
			
			<void-type> ::= \textbf{void}
						
			<constant> ::= <integral-literal>
			\alt <floating-literal>
			\alt <string-literal>
			\alt <boolean-literal>
			
			<integral-literal> ::= <digit>*
			
			<floating-literal> ::= <digit>* \textbf{.} <digit>*
			
			<string-literal> ::= \textbf{\"} ( "\\x00000000-\\x0010FFFF" )*  \textbf{\"}
			
			<boolean-literal> ::= \textbf{true}
			\alt \textbf{false}
			
			<alpha> ::= \textbf{a} | \textbf{b} | ... | \textbf{z} | \textbf{_}

			<digit> ::= \textbf{0} | \textbf{1} | ... | \textbf{9}

	
			<id> ::= <alpha> ( <alpha> | <digit> )*
			
			<parameter> ::= <type> <id>
			
			<parameter-list> ::= <parameter> \textbf{,} <parameter-list>
			\alt <parameter>
			
			<parameter-list-opt> ::= <parameter-list> | $\epsilon$
			
			<stmt> ::= <selection-stmt>
			\alt <iteration-stmt>
			\alt <jump-stmt>
			\alt <expr>
			
			<iteration-stmt> ::= <stmt>
			\alt \textbf{break};
			\alt \textbf{continue};
			
			<selection-stmt> ::= \textbf{if} \textbf{(} <expr> \textbf{)} \textbf{\{} <stmt>* \textbf{\}}
			\alt \textbf{if} \textbf{(} <expr> \textbf{)} \textbf{\{} <stmt>* \textbf{\}} \textbf{else} \textbf{\{} <stmt>* \textbf{\}}
			
			<iteration-stmt> ::= \textbf{for} \textbf{(} <expr-opt> \textbf{;} <expr-opt> \textbf{;} <expr-opt> \textbf{)} \textbf{\{} <iteration-stmt>* \textbf{\}}
			\alt \textbf{while} \textbf{(} <expr> \textbf{)} \textbf{\{} <iteration-stmt>* \textbf{\}}
			\alt \textbf{do} \textbf{\{} <iteration-stmt>* \textbf{\}} \textbf{while} \textbf{(} <expr> \textbf{)} 
			
			<jump-stmt> ::= \textbf{return} <expr>? \textbf{;}
			
			<assignment-op> ::= \tt{"="}
			\alt \tt{"*="}
			\alt \tt{"/="}
			\alt \tt{"%="}
			\alt \tt{"+="}
			\alt \tt{"-="}
			\alt \tt{"<<="}
			\alt \tt{">>="}
			\alt \tt{"&="}
			\alt \tt{"|="}
			\alt \tt{"^="}
			
			<expr> ::= <assignment-expr>
			
			<expr-opt> ::= <expr> | $\epsilon$
			
			<assignment-expr> ::= <logical-or-expr>
			\alt <unary-expr> <assignment-op> <assignment-expr>
			
			<logical-or-expr> ::= <logical-and-expr>
			\alt <logical-or-expr> \textbf{||} <logical-and-expr>
			
			<logical-and-expr> ::= <inclusive-or-expr>
			\alt <logical-and-expr> \textbf{\&\&} <inclusive-or-expr>
			
			<inclusive-or-expr> ::= <exclusive-or-expr>
			\alt <inclusive-or-expr> \textbf{|} <exclusive-or-expr>
			
			<exclusive-or-expr> ::= <and-expr>
			\alt <exclusive-or-expr> \textbf{\^} <and-expr>
			
			<and-expr> ::= <equality-expr>
			\alt <and-expr> \textbf{\&} <equality-expr>
			
			<equality-expr> ::= <relational-expr>
			\alt <equality-expr> \textbf{==} <relational-expr>
			\alt <equality-expr> \textbf{!=} <relational-expr>
			
			<relational-expr> ::= <shift-expr>
			\alt <relational-expr> \tt{">"} <shift-expr>
			\alt <relational-expr> \tt{"<"} <shift-expr>
			\alt <relational-expr> \tt{">="} <shift-expr>
			\alt <relational-expr> \tt{"<="} <shift-expr>
			
			<shift-expr> ::= <additive-expr>
			\alt <shift-expr> \tt{"<<"} <additive-expr>
			\alt <shift-expr> \tt{">>"} <additive-expr>
			
			<additive-expr> ::= <multiplicative-expr>
			\alt <additive-expr> \tt{"+"} <multiplicative-expr>
			\alt <additive-expr> \tt{"-"} <multiplicative-expr>
			
			<multiplicative-expr> ::= <unary-expr>
			\alt <multiplicative-expr> \tt{"*"} <unary-expr>
			\alt <multiplicative-expr> \tt{"/"} <unary-expr>
			\alt <multiplicative-expr> \tt{"%"} <unary-expr>
			
			<unary-expr> ::= <postfix-expr>
			\alt \tt{"++"} <unary-expr>
			\alt \tt{"--"} <unary-expr>
			
			<postfix-expr> ::= <primary-expr>
			\alt <postfix-expr> \tt{"["} <expr> \tt{"]"}
			\alt <postfix-expr> \tt{"++"}
			\alt <postfix-expr> \tt{"--"}
			
			<primary-expr> ::= <constant>
			\alt <id>
			\alt \tt{"("} <expr> \tt{")"}
			
		\end{grammar}

	\section{Environment}
		\upshape
		
		\subsection{Data types}
			The language must implement static strong typing. All casts must
			be explicit.
			\begin{itemize}
				\item \textbf{Int} -- Signed 32-bit;
				\item \textbf{Float} -- Signed 32-bit;
				\item \textbf{Bool} -- 8-bit;
				\item \textbf{String} -- Character sequence, that ends with Null character;
				\item \textbf{Void} -- Empty type, used as return type only.
			\end{itemize}
			
		\subsection{Inside-iteration statements}
			\begin{itemize}
				\item \textbf{Break} -- Usable only inside the \textbf{while},
		  			\textbf{do-while} and \textbf{for} statements and performs exit from a
			  		loop.
				
				\item \textbf{Continue} -- Usable only inside the \textbf{while},
		  			\textbf{do-while} and \textbf{for} statements and performs jump to
		  			the next iteration.
			\end{itemize}
			
		\subsection{Iteration statements}
			\begin{itemize}
		  		\item \textbf{While} -- Loop statement that performs its body until the
		  			condition evaluates to true.
		  		
		  		\item \textbf{Do-While} -- Loop statement with similar to \textbf{While}
		  			semantics, but it executes body before contition check at first time.
		  			
	  			\item \textbf{For} -- Loop statement with three initial parts and body.
	  				This includes:
	  				\begin{itemize}
	  					\item \textbf{Initial} part with the variable assignment;
	  					\item \textbf{Conditional} part with the some condition;
	  					\item \textbf{Incremental} part with the some statement, that
	  						should change assigned variable.
	  				\end{itemize}
			\end{itemize}
			
		\subsection{Conditional statements}
			\begin{itemize}
				\item \textbf{If} -- Conditional statement, that should execute
				If-part when it's condition evaluates to true. Otherwise, Else-part
				should be executed. 
			\end{itemize}
			
		\subsection{Jump statements}
			\begin{itemize}
				\item \textbf{Return} -- The end point of control flow, may return
				value, may not (void functions).
			\end{itemize}
			
		\subsection{Translation environment}
			The whole program must be placed in one file to simplify translation
			and linking (lack of it as such).
			
	\section{Implementation}
	
		\subsection{Intermediate representation format}
			The IR has three-address format, which consists of following instructions:
			
			\begin{itemize}
				\item \textit{\textbf{res} = \textbf{arg} op \textbf{arg}};
				
				\item \textit{\textbf{if} \textbf{arg} op \textbf{arg} goto 
				\textbf{L}};

				\item \textit{\textbf{L}:} (label);

				\item \textit{goto \textbf{L}}.
			\end{itemize}
			
			\noindent \textbf{NOTE:} \textbf{\textit{arg}} can be either temporary variable
			or immediate value (integer or float).
				
			\bigbreak
				
			\noindent \textbf{NOTE:} \textbf{\textit{if}} instruction explicitly has two 
			operands, so if	we have expression like \[\textbf{\textit{if (condition)}}\],
			 \textbf{\textit{if}} instruction should be represented as
			 \[\textbf{\textit{if condition != 0 goto L...}}\].
			 
			 \bigbreak
			 
			 \noindent \textbf{NOTE:} when \textbf{\textit{if}} instruction argument
			 cannot explicitly be converted to bool, the result of argument implicitly
			 converted to bool.
			
\end{document}