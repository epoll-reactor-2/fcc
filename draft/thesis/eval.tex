\section{Interpreter}

	\begin{center}
		\begin{tabularx}{\textwidth}{ L{9cm} L{9cm} }
			% https://tex.stackexchange.com/questions/235000/drawing-an-activation-stack-in-latex
			% https://lpn-doc-sphinx-primer.readthedocs.io/en/latest/extensions/tikz/tikzgoodies.html
			\begin{tikzpicture}[x=1.05cm,y=0.7cm]
				\small

				\cell[draw=none]{Stos}
				\cell[fill=white]{\texttt{call main()}}     \cellcomL{global=0}          \coordinate (p0)    at (currentcell.east);
				\cell[fill=white]{\texttt{int a = 0}}       \cellcomL{global=4}          \coordinate (p4)    at (currentcell.east);
				\cell[fill=white]{\texttt{int b = 0}}       \cellcomL{global=8}          \coordinate (p8n1)  at (currentcell.east);
				\cell[fill=white]{\texttt{call f()}}        \cellcomL{global=8}          \coordinate (p8n2)  at (currentcell.east);
				\cell[fill=white]{\texttt{int c = 0}}       \cellcomL{global=12,local=4} \coordinate (p12)   at (currentcell.east);
				\cell[fill=white]{\texttt{int d = 0}}       \cellcomL{global=16,local=8} \coordinate (p16n1) at (currentcell.east);
				\cell[fill=white]{\texttt{return from f()}} \cellcomL{global=8}          \coordinate (p16n2) at (currentcell.east);
				\cell[fill=white]{\texttt{return b}}        \cellcomL{global=8}          \coordinate (p16n3) at (currentcell.east);
				\stackbottom[fill=white]{}

				% todo: Draw from borders.
				\draw[->] (p16n3) to [bend right=100] (p8n1);
			\end{tikzpicture}
			&
			Interpreter -- program, który produkuje dane wyjściowe zgodnie z zasadami semantycznymi, które
			są opisane językiem. W naszym przypadku, interpreter przetwarza wygenerowany poprzednio IR.
			$$$$
			Niniejszy interpreter działa używając stosu. Jest to struktura danych, pozwalająca dodawać i 
			usuwać dane tylko z górnej części stosu. Określamy operacje \textbf{push} i \textbf{pop}.
			Stos, używany przez interpreter jednak wspiera przegląd wartości o dowolnej pozycji w stosie.
		\end{tabularx}
	\end{center}
	
	Aby zarządzać pamięcią podczas wykonania, musimy zdecydować, na jaki sposób zaimplementować
	\textbf{adresację} zmiennych. Jest to operacja, która utożsamia nazwę zmiennej z jej adresem w pamięci.
	Istineije dwie możliości
	\\
	
	\textbf{Statyczna alokacja pamięci} -- sposób adresacji, przy którym adres każdej zmiennej określony
	jednoznacznie. Jest użyteczna do początkowych wersji ALGOL i COBOL, które nie wspierają rekurencyjne
	wywołania funckji.
	\\
	
	\textbf{Dynamiczna alokacja pamięci} -- spobób adresacji, przy którym niemożliwe jest obliczenie
	fizycznego adresu zmienneij, skoro stworzona ona może być w funkcji, wywołanej przez inną funkcję
	bądź samą siebie. Naprzykład, dana funkcja \textbf{f()}, która definiuje zmienną \textbf{i}, i
	załóżmy, że wywoła ona się rekurencyjnie. Wtedy podczas pierwszego wywołania zmienna \textbf{i}
	będzie miała adres \textbf{0x00}. Podczas drugiego wywołania \textbf{0x04}, dalej \textbf{0x08},
	\textbf{0x0C}, ...
