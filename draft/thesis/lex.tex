\section{Analiza leksykalna}

	Jednym ze sposobów na sprowadzanie kodu źródłowego do postaći listy tokenów jest narzędzie flex.
	Przyjmuje ono zestaw reguł w postaći wyrażeń regularnych, według których
	działa rozbicie tekstu wejściowego. Można jednak ominąć lex i zaimplemenetować lexer ręcznie, ale
	ta praca nie skupia się na tym.

	\subsection{Wyrażenia regularne}

		Wyrażenie regularne -- łańcuch znaków, zawierający pewne polecenia do wyszukiwania tekstu.
		\\

		Mówimy, że wyrażenie regularne określone nad alfabetem $\Sigma$, jeżeli zachodzą następujące
		warunki:

		\begin{itemize}
			\item $\emptyset$ -- wyrażenie regularne, reprezentujące pusty zbiór.
			\item $\epsilon$ -- wyrażenie regularne, reprezentujące pusty łańcuch.
			\item $\forall_{a \in \Sigma}$, $a$ reprezentuje jeden znak.
			\item Warunek indukcyjny: jeżeli $R_1, R_2$ -- wyrażenia regularne, $(R_1 R_2)$
			      stanowi konkatenację $R_1$ i $R_2$.
	      	\item Warunek indukcyjny: jeżeli $R$ -- wyrażenie regularne, $R*$ stanowi domknięcie Kleene'ego. 
		\end{itemize}

		\spacing

		W rzeczywistości, takich zasad może być więcej.
		\\

		Zazwyczaj wyrażenie regularne jest realizowane za pomocą DFA (Deterministic finite automaton,
		Deterministyczny automat skończony).
		Lex sprowadza podany zbiór zasad do takiego automatu.
		\\

		Podamy przykład automatu dla wyrażenia \texttt{-?[0-9]+}
		
		\spacing
		\spacing

		\begin{center}
		\begin{tikzpicture}[node distance=4cm]
			\node[state, initial, accepting] (0) {0};
			\node[state, right of=0]         (1) {1};
			\node[state, right of=1]         (2) {2};
			\draw
			(0) edge[       bend left ]             (1)
			(0) edge[below, bend right] node{$-$}   (1)
			(1) edge[below            ] node{$0-9$} (2)
			(2) edge[loop below       ] node{$0-9$} (2)
			;
		\end{tikzpicture}
		\end{center}
		
		Aby odśledzić wykonane kroki, można wypełnić tabelę przejść pomiędzy stanami. Podamy przykład dla
		łańcucha \texttt{-22}

		\spacing
		\spacing

	\begin{center}
		\setlength{\tabcolsep}{0.5em}
		\renewcommand{\arraystretch}{1.2}
		\begin{tabular}{ | L{3cm} | L{4cm} | }
			\hline
			Biężący stan        & Akcja \\
			\hline
			\textbf{0}          & zaakceptować - \\
			\hline
			0, \textbf{1}       & zaakceptować 2 \\
			\hline
			0, 1, \textbf{2}    & zaakceptować 2 \\
			\hline
		\end{tabular}
	\end{center}
	
	\newpage

\subsection{Flex}

	Flex jest narzędziem projektu GNU. Pozwala ono w wygodny sposób podać listę reguł dla analizy
	leksykalnej (ang. Scanning). Flex jest mocno powiązany z językiem C, dlatego program w flex'u
	korzysta z konstrukcji języka C. Pokażemy przykład użycia flex'u
	
	\spacing
	
\begin{lstlisting}
%{
#include "portrzebny-do-analizy-plik.h"

/* Kod w jezyku C. */
%}

/* Opcje flex */
%option noyywrap nounput noinput
%option yylineno

%% /* Reguly w postaci wyrazen regularnych. */

/*********************************************************************/
/* Wzorzec                   | Akcja przy znalezieniu takiego wzorcu */
/*********************************************************************/
-?[0-9]+                       LEX_CONSUME_WORD(TOK_INTEGRAL_LITERAL)
-?[0-9]+\.[0-9]+               LEX_CONSUME_WORD(TOK_FLOATING_POINT_LITERAL)
\"([^\"\\]*(\\.[^\"\\]*)*)\"   LEX_CONSUME_WORD(TOK_STRING_LITERAL)
\'.\'                          LEX_CONSUME_WORD(TOK_CHAR_LITERAL)

.                              { /* Znaleziony niewiadomy znak.
                                    Zglosic blad.
                                  */ }
%%
\end{lstlisting}

	\spacing
	
	Zauważmy, że flex próbuje szukać wzorców w tekscie dokładnie w takiej kolejności, która jest podana w
	jego kodzie. Dlatego często robią ostatnią regułe z wyrażeniem regularnym ".", który akceptuje
	dowolny znak, i umieszczają tam komunikat o błędzie.
	\\

	W naszym przypadku, lex generuje kod, który gromadzi wszystkie znalezione lexemy do tablicy.

	\newpage
