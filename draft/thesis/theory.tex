\section{Teoria}

	\subsection{Języki formalne}

		Według teorii automatów, automat -- jest to jednostka wykonawcza. Jednostki te, zależnie od swojej
		struktury i tego, jaki \textbf{język formalny} oni mogą obrobić, dzielą się na klasy.
		\\
		
		Klasy te opisane są \textbf{hierarchią Chomsky’ego}. Mówi ona o tym, że języki formalne dzielą się na
		4 typy:
		\begin{itemize}
			\item Typ 3 -- języki regularne
			\item Typ 2 -- języki bezkontekstowe
			\item Typ 1 -- języki kontekstowe
			\item Typ 0 -- języki rekurencyjnie przeliczalne
		\end{itemize}
		
		\spacing

		Jako przykład języka typu 3 według hierarchii Chomsky'ego można podać wyrażenia regularne. Język ten
		opisuje się automatem skończonym deterministycznym (DFA). Bardziej szczegółowo wyrażenia regularne będą
		rozpatrzone w opisaniu analizy leksykalnej.
		
	\subsection{Klasyfikacja gramatyczna}

		Niniejszy język nie może być odniesiony do żadnej z klas hierarchii Chomsky'ego, chociaż jest on
		językiem regularnym. Tak jest dlatego, że można napisać gramatycznie poprawny kod, który jednak prowadzi
		do błędów kontekstowych i logicznych. Naprzykład 
		
		\spacing
		
\begin{lstlisting}[caption={}, label={lst:ambigous-production}]
void f() {
	return argument + 1;
}
\end{lstlisting}

		\spacing
		
		Kolejną z przyczyn niemożliwości odniesienia naszego języka do jednej z klas hierarchii Chomsky'ego
		jest niejednoznaczność konstrukcji językowych. Przykład niżej pokazuje, że nie można jednoznacznie
		stwierdzić, czy \texttt{data * d} jest deklaracją zmiennej albo operatorem mnożenia dwóch zmiennych.
		Aby móc poprawnie prowadzić analizę składniową, musimy zadbać o rozróżnienie kontekstu.

		\spacing
		
\begin{lstlisting}[caption={}, label={lst:ambigous-production}]
void f() {
	data *d;
}
\end{lstlisting}
