\section{Optymalizacje (teoria)}
	
	Optymalizacja -- to zmiana kodu programu, mająca na celu polepszyć wydajność albo inne cechy
	programu. Najważniejym z kriteria optymalizacji jest utrzymanie całej struktury działania programu,
	takiej, jak chce programista. Nie wolno przeprowadzać wiodące do niespodziewanych lub
	niepoprawnych wyników optymalizacje.
	\\
	
	Istnieje wiele rodzajów optymalizacji, gdzie każda wymaga więcej lub mniej założeń i
	matematyki. Jeżeli chodzi o matematykę, to główną rolę w optymalizacji pełni \textbf{teoria grafów}.
	Jednym z pierwszych naukowców, kto zadecydował wprowadzić modele grafowe do kompilatorów był
	\textbf{Robert Tarjan}. Wprowadził także on algorytm do obliczenia drzewa dominatorów (dominator tree).

	\subsection{Definicje}
		Każdy program posiadą taką cechę jak \textbf{przepływ sterowania}, i może ona być
		precyzyjnie wyrażona \textbf{grafem przepływu sterowania} (Control Flow Graph). Program
		rozpatrywany jest jako graf skierowany, posiadający wierzchołek startowy, będący pierwszą
		instrukcją w programie. Krawędzie reprezentują przejścia pomiędzy instrukcjami. Zauważmy,
		że każda instrukcja może mieć wiele krawędzi wejściowych i wyjściowych, tworząc \textbf{gałęzi}
		w wykonaniu programu.
		\\
		
		Oznaczmy kilka ważnych pojęć.
		\\

		Niech $G = (V, E, s)$ -- graf skierowany.
		$V$ -- zbiór wierzchołków. $E$ -- zbiór krawędzi. $s$ -- węzęł początkowy.
		$G' = (V', E') \subseteq G$ -- podgraf, gdzie każdy wierzchołek $v \in G'$ jest nieosiągalny z
		dowolnej ścieżki $(s, ...,  v) \in G$. Zauważmy, że $G'$ może być grafem rozłącznym.

		$$V' = \{ \ v \ | \ \forall \ v \ \nexists \ (s, ..., v) \ \}$$

		\spacing

		Niech $G = (V, E, s)$ -- graf skierowany, zdefiniowany powyżej. Graf zależności danych -- graf
		$G' = (V', E', D')$. Wtedy $D' = \{ \ d, d' \in V' \ | \ d \rightarrow d' \ \}$. Przez $d \rightarrow d'$
		oznaczona zależnośc $d$ od wyniku obliczenia $d'$. Zbiór $D'$ reprezentuje takie zależności, przy czym
		dodatkowo musi być spełnione

		$$d, d' \in V', \ v \rightarrow v' \iff \exists \ (v', ..., v) \in G' $$
		
		Kilka ważnych oznaczeń
		
	    \begin{center}
		    \setlength{\tabcolsep}{0.5em}
		    \renewcommand{\arraystretch}{1.5}
		    \definecolor{instrc}{gray}{0.4}
		    \begin{tabular}{| L{0.2\linewidth} | L{0.75\linewidth} | }
			    \hline
			    $X \rightsquigarrow Y$ & X jest dominatorem Y \\
			    \hline
			    $X \not{\rightsquigarrow Y}$ & X \textbf{nie} jest dominatorem Y \\
			    \hline
			    $Pred(X)$ & Poprzednik X \\
			    \hline
			    $Succ(X)$ & Następca X \\
			    \hline
		    \end{tabular}
	    \end{center}

		\subsection{SSA}
	Static Assignment Form (SSA) -- ważna forma reprezentacji programu. Nazywa się
	tak program, w którym każda zmienna jest zapisana dokładnie jeden raz. Odkrywa
	to wiele możliwości do prowadzenia optymalizacji, na przykład jest to propagacja
	 zmiennych stałych i alokacja rejestrów.

	Konstrukcja SSA składa się z kilku etapów:
	\begin{itemize}
		\item Obliczenie drzewa dominatorów (dominator tree).
		\item Obliczenie granicy dominacji (dominance frontier).
		\item Wstawianie $\phi$-funkcji (zob. niżej).
	\end{itemize}
	Istnieje przynajmniej dwa algorytmy do obliczenia SSA na różne sposoby. Pierwszy
	autorstwa \textbf{Ron Cytron} i kilku innych matematyków (An Efficient
	Method of Computing Static Single Assignment Form). Drugi autorstwa
	\textbf{Quan Hoang Nguyen} (Computing SSA Form with Matrices). Został
	zaimplementowany pierwszy algorytm, skoro jest prostrzy i wymaga mniej (chociaż
	też dużo) kultury matematycznej. Nie będziemy udowadniać tutaj całego algorytmu,
	pokażemy tylko główne jego idee.
	
	\subsection{SSA: drzewo dominatorów}
		Do obliczenia drzewa dominatorów używa się algorytm, stworzony przez dwóch
		matematyków -- algorytm \textbf{Lengauer-Tarjan}'u. Jest on profesjonalnie
		zbudowany i udowodniony matematycznie. Składa się on z trzech części
		\begin{enumerate}
			\item Przejście grafu w głąb (zwykły DFS).
			\item Przejście wyniku DFS w odwrotnej kolejności, zatem przeliczanie
			      półdominatorów (\textbf{semidominators}).
	      	\item Za pomocą półdominatorów definicja dominatorów.
		\end{enumerate}

		\spacing
		
		Dominator węzła \textit{w} -- taki węzeł \textit{v}, który występuje w każdej ścieżce od
		węzła początkowego grafu (w tym przypadku) do \textit{w}. Oznacza się $v \ doms \ w$.
		\\

		Półdominator węzła \textit{w} -- taki węzeł, który jest połączony z \textit{w} i
		każdy węzeł po drodze do \textit{w} jest \textbf{większy} od \textit{w}. Pod słowem
		\textbf{większy} lub \textbf{mniejszy} węzeł rozumiemy relację odnośnie kolei przejścia
		węzłów w głąb (DFS). Kolejność DFS dla podanego niżej grafu: \textbf{2, 1, 4, 3}.

		\spacing
	
		\begin{center}
		\begin{tabular}{ R{9cm} L{9cm} }
			\begin{tikzpicture}[->,>=stealth',shorten >=1pt,auto,node distance=2.5cm]

			  \node (1) {1};
			  \node (2) [right of=1] {2};
			  \node (3) [below of=2] {3};
			  \node (4) [below of=1] {4};

			  \path[every node/.style={font=\sffamily\small}]
				(2) edge node [left] {} (1)
				(2) edge node [right] {} (3)
				(2) edge node [right] {} (4);

				\draw [arrow, bend left=45] (2.south) to (3.east);
				\draw [arrow              ] (4.north) to (1.south);
			\end{tikzpicture}
		&
			\begin{tikzpicture}[->,>=stealth',shorten >=1pt,auto,node distance=2.5cm]

			  \node (1) [] {2};
			  \node (2) [below of=1] {1};
			  \node (3) [below of=1, left of=1] {3};
			  \node (4) [below of=1, right of=1] {4};
			  
			  \draw [arrow] (1) to (2);
			  \draw [arrow] (1) to (3);
			  \draw [arrow] (1) to (4);
			\end{tikzpicture}
		\end{tabular}
		\end{center}
		
		% Wierd as fuck.
		\begin{tabular}{ R{9cm} L{9cm} }
			Graf \ \ \ \ \ \ \ \ \ \ \ \ & \ \ \ \ \ Jego drzewo dominatorów
		\end{tabular}

		\spacing

		Tak się formalnie określa półdominator:

		\begin{equation*}
		\begin{split}
			semidom(w) = min(& \{ \\
						     & \ \ \ \ v \ | \ (v, w) \ \in \ E \ \ \&\& \ \ v \ < \ w \\
					         & \} \ \cup \ \{ \\
					         & \ \ \ \ semidom(u) \ | \ u \ > \ w \ \ \&\& \ \ E \ is \ edge \ (v, w) \\
					         & \})
		\end{split}
		\end{equation*}
		
	\subsection{SSA: granica dominacji}
	
	    Granica dominacji -- cecha grafu skierowanego, która powstała głównie do stworzenia algorytmu
	    do obliczania SSA od Rona Cytrona. Granicą węzła jest taki zbiór węzłów, na których
	    "relacja dominacji tego węzła nad innymi się kończy". Używamy algorytmu Keith D. Cooper'u.
	    \\
	    
	    Formalna definicja, wzięta z algorytmu SSA:

		\begin{equation*}
		\begin{split}
			DF(X) = \{ Y \ | \ (\exists P \in Pred(Y)) (X \gg P \ and \not{\gg} Y) \}
		\end{split}
		\end{equation*}

        Algorytm do obliczenia granicy dominacji polega na przejściu drzewa dominatorów w górę.
	
		\begin{algorithm}
			\caption{Dominance Frontier}
			\begin{algorithmic}[1]

			\Procedure{DF}{Node}
			    \If{Preds(Node) $>=$ 2}
			        \For{each predecessor P of Node}
			        \State Runner $\gets$ P
			        \While{P != Doms(Node)}
			            \State Add Node to Runner's DF set
			            \State Runner $\gets$ Doms(Runner)
			        \EndWhile
			        \EndFor
			    \EndIf
			\EndProcedure

			\end{algorithmic}
		\end{algorithm}

	\subsection{SSA: $\phi$-funkcje}
