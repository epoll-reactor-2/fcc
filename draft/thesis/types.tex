\section{System typów}

	\subsection{Opis}

		Wiele zasad, dotyczących pracy z typami mogą być precyzyjnie opisane zasadami typów
		(\textbf{Typing rules}). Jest to notacja matematyczna, stworzona przez \textbf{Per Martin-Löf}'a,
		szwedzkiego matematyka.
		\\

		Kluczowym pojęciem w tej notacji jest \textbf{statyczne środowisko typów}
		(\textbf{static typing environment}). Oznacza się ono symbolem $\Gamma$. Mówimy, że to
		środowisko jest skonstruowane poprawnie pisząc $$\Gamma \vdash \diamond$$
		
		Mówimy, że zmienna \textbf{V} ma typ \textbf{T} w środowisku $\Gamma$ pisząc
		$$\Gamma \vdash \text{V} : T$$
		
		Kreska pozioma mówi o tym, że zdanie wyżej jest konieczne, aby zaszło zdanie niżej

		\begin{mathpar}
		\inferrule
		  {\Gamma \vdash \diamond}
		  {\Gamma \vdash \text{V} : T}
		\end{mathpar}
		
		Zauważmy, że notacja ta jest mocnyn narzędziem, pozwalającym opisać dość złożone systemy typów
		dla takich języków jak \textbf{C++} i \textbf{Haskell}.

	\subsection{Definicja}

		Opiszmy teraz system typów w naszym języku

		\begin{mathpar}
		\inferrule
		  {\Gamma \vdash \diamond}
		  {\Gamma \vdash \text{true} : bool}
		\quad

		\inferrule
		  {\Gamma \vdash \diamond}
		  {\Gamma \vdash \text{false} : bool}
		\end{mathpar}

		\begin{mathpar}
		\inferrule
		  {\Gamma \vdash \diamond}
		  {\Gamma \vdash \text{n} : int}

		\inferrule
		  {\Gamma \vdash \diamond}
		  {\Gamma \vdash \text{c} : char}

		\inferrule
		  {\Gamma \vdash \diamond}
		  {\Gamma \vdash \text{x} : float}
		\end{mathpar}

	Oznaczmy dla $\mathbb{N, R}: \oplus \in \{ =, +, -, *, /, <, >, \le, \ge, ==, \neq, ||, \&\& \}$, wtedy

		\begin{mathpar}
		\inferrule
		  {\Gamma \vdash e_l : float \\ \Gamma \vdash e_r : float}
		  {\Gamma \vdash e_l \oplus e_r : float}
		\end{mathpar}

		\spacing

	Dodamy do $\oplus$ operacje tylko dla $\mathbb{N}: \oplus \ \cup \ \{ |, \&, \string^, <<, >>, \% \}$, wtedy

		\spacing

		\begin{mathpar}
		\inferrule
		  {\Gamma \vdash e_l : int \\ \Gamma \vdash e_r : int}
		  {\Gamma \vdash e_l \oplus e_r : int}

		\inferrule
		  {\Gamma \vdash e_l : int \\ \Gamma \vdash e_r : char}
		  {\Gamma \vdash e_l \oplus e_r : char}
		\end{mathpar}

		\spacing

		Wprowadźmy reguły niejawnej konwersji, które są niezbędne przy sprawdzaniu w warunku
		logicznym wyniku operacji arytmetycznej, zwracającej typ różny od
		\texttt{bool}. Oznaczmy reguły dla typów \texttt{int}, \texttt{char} i \texttt{float}.

		\begin{mathpar}
		\inferrule
			{\Gamma \vdash \text{e} : int}
			{\Gamma \vdash \text{e} : bool}

		\inferrule
			{\Gamma \vdash \text{e} : char}
			{\Gamma \vdash \text{e} : bool}

		\inferrule
			{\Gamma \vdash \text{e} : float}
			{\Gamma \vdash \text{e} : bool}
		\end{mathpar}

		\spacing
		
		Oznaczmy konwersję każdego typu z każdym

		\begin{mathpar}
			\inferrule
				{\Gamma \vdash \text{e} : int}
				{\Gamma \vdash \text{e} : float}

			\inferrule
				{\Gamma \vdash \text{e} : int}
				{\Gamma \vdash \text{e} : char}

			\inferrule
				{\Gamma \vdash \text{e} : int}
				{\Gamma \vdash \text{e} : bool}

			\inferrule
				{\Gamma \vdash \text{e} : float}
				{\Gamma \vdash \text{e} : int}

			\inferrule
				{\Gamma \vdash \text{e} : float}
				{\Gamma \vdash \text{e} : char}

			\inferrule
				{\Gamma \vdash \text{e} : float}
				{\Gamma \vdash \text{e} : bool}

			\inferrule
				{\Gamma \vdash \text{e} : char}
				{\Gamma \vdash \text{e} : int}

			\inferrule
				{\Gamma \vdash \text{e} : char}
				{\Gamma \vdash \text{e} : float}

			\inferrule
				{\Gamma \vdash \text{e} : char}
				{\Gamma \vdash \text{e} : bool}

			\inferrule
				{\Gamma \vdash \text{e} : bool}
				{\Gamma \vdash \text{e} : float}

			\inferrule
				{\Gamma \vdash \text{e} : bool}
				{\Gamma \vdash \text{e} : int}

			\inferrule
				{\Gamma \vdash \text{e} : bool}
				{\Gamma \vdash \text{e} : char}
		\end{mathpar}

		\spacing

		Wprowadźmy także reguły do operacji wskaźnikowych. Oznaczmy
		$\oplus \in \{ +, - \}$ i $\tau$ jako wskaźnik dowolnego typu, wtedy

		\begin{mathpar}
		\inferrule
			{\Gamma \vdash e_l : \tau \ * \\ \Gamma \vdash e_r : \tau \ *}
			{\Gamma \vdash e_l \oplus e_r : \tau \ *}

		\inferrule
			{\Gamma \vdash e_l : \tau \ * \\ \Gamma \vdash e_r : int}
			{\Gamma \vdash e_l \oplus e_r : \tau \ *}

		\inferrule
			{\Gamma \vdash e_l : int \\ \Gamma \vdash e_r : \tau \ *}
			{\Gamma \vdash e_l \oplus e_r : \tau \ *}
		\end{mathpar}

		Dla $\oplus \in \{ ==, !=, <, >, \le, \ge \}$

		\begin{mathpar}
		\inferrule
			{\Gamma \vdash e_l : \tau \ * \\ \Gamma \vdash e_r : \tau \ *}
			{\Gamma \vdash e_l \oplus e_r : int}
		\end{mathpar}

		Mając taką konwersję, możemy wprowadzić reguły do konstrukcji warunkowych: 

		\begin{mathpar}
		\inferrule
		  {\Gamma \vdash \text{condition} : bool \\ \Gamma \vdash e_1 : \tau \\ \Gamma \vdash e_2 : \tau}
		  {\Gamma \vdash \text{if (condition) } \{ \ e_1 \ \}  \text{ else }  \{ \ e_2 \ \} : \tau}

		\inferrule
			{\Gamma \vdash \text{condition} : bool \\ \Gamma \vdash e : \tau}
			{\Gamma \vdash \text{while (condition) } \{ \ e \ \} : \tau}

		\inferrule
			{\Gamma \vdash \text{condition} : bool \\ \Gamma \vdash e : \tau}
			{\Gamma \vdash \text{do } \{ \ e \ \} \text{ while (condition)} : \tau}

		\inferrule
			{\Gamma \vdash \text{init} : \tau_1 \\
			 \Gamma \vdash \text{condition} : bool \\
			 \Gamma \vdash \text{increment} : \tau_2 \\ \Gamma \vdash e : \tau_3}
			{\Gamma \vdash \text{for (init; condition; increment) } \{ \ e \ \} : \tau_3}
		\end{mathpar}

	\subsection{Implementacja}
		
		Implementacja systemu typów w niniejszym języku polega na dwóch krokach
		
		\begin{itemize}
			\item Dodanie konwersji nejawnych. Implementuje to funkcja \texttt{sema\_type()}.
			\item Sprawdzenie, czy po dodaniu konwersji wszystkie operacje na typach zgodne z
			      definicją.
		\end{itemize}