\section{Analiza składniowa}
	
	\subsection{Definicja}

		Mając listę składników elementarnych wejściowego programu, jesteśmy w stanie przejść do
		następnego etapu kompilacji -- analizy składniowej. Jest to proces generacji struktury
		drzewiastej, a mianowicie AST (Abstract Syntax Tree).
		\spacing
		
		Wynikiem działania analizy składniowej zawsze jest \textbf{jedno} drzewo AST. Może zawieraić ono
		definicje wszystkich funkcji.
		\\
	
		AST może być stworzony po zdefiniowaniu gramatyki regularnej danego języka. Stosuje się
		do tego notacja BNF (Backus–Naur form). Pełny opis gramatyki pokazany jest w końcu pracy. Pokażemy
		tylko kilka przykładów:
		
		\spacing

		\setlength{\grammarindent}{12em}

		\begin{grammar}
		    <program> ::= ( <function-decl> | <structure-decl> )*

		    <var-decl> ::= <type> ( \tt{*} )* <id> \tt{=} <logical-or-stmt> \tt{;}

		    <stmt> ::= <block-stmt>
		    \alt <selection-stmt>
		    \alt <iteration-stmt>
		    \alt <jump-stmt>
		    \alt <decl>
		    \alt <expr>
		    \alt <assignment-stmt>
		    \alt <primary-stmt>
		\end{grammar}
		
		\spacing
		
		W przypadku wyrażeń arytmetycznych, AST także jednoznacznie określa za pomocą produkcji
		gramatyki BNF priorytet operacji
		arytmetycznych. Naprzykład, mając wyrażenie \ \texttt{1 + 2 * 3 + 4}, drzewo syntaksyczne
		będzie skonstruowane zgodnie z prawami arytmetyki, co pozwala nie trzymać w AST żadnych
		informacji o nawiasach. Widać, że aby zastosować produkcję \textit{$<$additive-stmt$>$},
		najpierw musi być zastosowana następna produkcja \textit{$<$multiplicative-stmt$>$}.
		\\

		Pomocnicza przy prowadzeniu analizy jest \textbf{tablica parsingu}. Jest to zbiór
		konkretnych przejść pomiędzy produkcjami. Pomaga ona w zrozumieniu, jaką
		produkcję zastosować mając dany nieterminal. Zauważmy, że w tabelę są wpisane produkcje bez
		alternatyw, i każde przejście gramatyczne określone jednoznacznie.
		\\
		
		Aby zbudować tą tablicy, możemy użyć zasady
		\textbf{First \& Follow}. Tutaj \textbf{First} to zbiór terminalnych symboli, które mogą pojawić
		się jako
		pierwsze w ciągu znaków wygenerowanym przez daną nieterminalną symbol w gramatyce,
		a \textbf{Follow} to
		zbiór terminalnych symboli, które mogą wystąpić bezpośrednio po danym nieterminalnym symbolu
		w dowolnym ciągu znaków wygenerowanym przez gramatykę.
		\\
		
		Podamy gramatykę dla przykładu powyżej (\texttt{1 + 2 * 3 + 4}). Musimy wprowadzić dwa poziomy
		priorytetów, aby prawidłowo zachować kolejność operacji mnożenia i dodawania.
		\\

		\begin{grammar}
		    <additive-stmt> ::= <multiplicative-stmt>
		    \alt <multiplicative-stmt> \tt{"+"} <additive-stmt>
		    \alt <multiplicative-stmt> \tt{"-"} <additive-stmt>

		    <multiplicative-stmt> ::= <prefix-unary-stmt>
		    \alt <prefix-unary-stmt> \tt{"*"} <multiplicative-stmt>
		    \alt <prefix-unary-stmt> \tt{"/"} <multiplicative-stmt>
		    \alt <prefix-unary-stmt> \tt{"%"} <multiplicative-stmt>
		\end{grammar}
		
		\spacing

		\begin{center}
			\setlength{\tabcolsep}{0.5em}
			\renewcommand{\arraystretch}{1.5}
			\begin{tabular}{ L{5cm} L{3cm} L{4cm} }
					                      & \textbf{First} & \textbf{Follow} \\
				$<$additive-stmt$>$       & 0-9            & \textbf{+}, \textbf{-} \\
				$<$multiplicative-stmt$>$ & 0-9            & \textbf{*}, \textbf{/} \\
				$<$prefix-unary-stmt$>$   & 0-9            & $\epsilon$ \\
			\end{tabular}
		\end{center}

		\spacing

		\begin{center}
			\setlength{\tabcolsep}{0.5em}
			\renewcommand{\arraystretch}{1.5}
			\begin{tabular}{ | L{4.1cm} | L{2cm} | L{2cm} | L{2cm} | L{2cm} | L{2cm} | L{2cm} | }
              	\hline
              	       & 0-9    & +    & -    & *    & /    & \$ \\
              	\hline
          	 	$<$additive-stmt$>$
          	 	       &        & $<$mul$>$ + $<$add$>$
          	 	                & $<$mul$>$ - $<$add$>$
          	 	                & & & $<$mul$>$ \\
              	\hline
              	$<$multiplicative-stmt$>$ 
              		   &        & & & $<$una$>$ * $<$mul$>$
              		                & $<$una$>$ / $<$mul$>$
              		                & $<$una$>$ \\
              	\hline
              	$<$prefix-unary-stmt$>$
              	       & 0-9 & & & & & \\
              	\hline
			\end{tabular}
		\end{center}

		\spacing

		\begin{center}
		\begin{tikzpicture}
			[level distance=1.5cm]
			\node[state] {+}
			child {
		  		node[state] {+}
		  		child { node[state] {\textbf{1}} }
		  		child {
		  			node[state] {*}
		  			child { node[state] {\textbf{2}} }
	  				child {node[state] {\textbf{3}} }
		  		}
	  		}
	  		child {
	  			node[state] {\textbf{4}}
	  		};
		\end{tikzpicture}
		\end{center}

	\subsection{Eliminacja rekurencji lewej}

		Projektując gramatykę, należy wziąć pod uwagę problem rekurencji lewej (Left recursion).
		Są produkcje gramatyczne, nie pozwalające kodu, które je implementuje przejść do następnego
		terminalu, stosując tą samą produkcję, co prowadzi do rekurencji nieskończonej.
		\\
		
		Rekuręcja lewa może wyglądać następująco:

		\begin{grammar}
			<factor> ::= <factor> '+' <term>
		\end{grammar}

		\spacing

		Kod, wykonujący tą regułe będzie miał postać:
		
		\spacing

\begin{lstlisting}[label={lst:left-recursion}]
void factor() {
	factor(); // Rekurencja bez zadnego warunku wyjscia
	consume('+');
	term();
}
\end{lstlisting}
		
		\subsection{Niejednoznaczność}
		
			Projektując język, łatwo trafić na niejednoznaczne produkcje gramatyczne. One są takie, że
			jednego tekstu wejściowego są one w stanie wyprodukować kilka różnych od siebie drzew. Popularny
			warunek dla stworzenia niejednoznaczności to taka produkcja
			
			\begin{grammar}
			    <P> ::= <P> + <P>
				\alt <symbol>
			\end{grammar}
			
			
			Po zastosowaniu danej produkcji dla \texttt{A + B + C} możliwe jest otrzymanie dwóch drzew
			
			\spacing
			\spacing

			\begin{center}
				\begin{tabular}{ L{5cm} L{5cm} }
					\begin{tikzpicture}
						[level distance=1.5cm]
						\node[state] {+}
						child {
					  		node[state] {+}
					  		child { node[state] {\textbf{A}} }
					  		child { node[state] {\textbf{B}} }
				  		}
				  		child {
				  			node[state] {\textbf{C}}
				  		};
					\end{tikzpicture}
					&
					\begin{tikzpicture}
						[level distance=1.5cm]
						\node[state] {+}
						child {
				  			node[state] {\textbf{A}}
				  		}
				  		child {
					  		node[state] {+}
					  		child { node[state] {\textbf{B}} }
					  		child { node[state] {\textbf{C}} }
				  		};
					\end{tikzpicture}
				\end{tabular}
			\end{center}
			
			Aby rozwiązać ten problem i jednoznacznie wskazać kolejność zastosowania reguł gramatycznych,
			możemy zamienić prawy operand na symbol, wtedy eliminuje się dwuznaczność. Pierwsza produkcja
			poniżej jest \textbf{lewostronną}, a druga -- \textbf{prawostronną}.

			\begin{grammar}
			    <P> ::= <P> + <symbol>
	
			    <P> ::= <symbol> + <P>
			\end{grammar}

		\subsection{Implementacja AST}

			Zaimplementowany AST składa się ze struktury \texttt{ast_node}. Jest to główny typ
			węzła, zawierający niektóre zbędne informacje dla każdego typu węzła AST, i przechowujący
			konkretny węzęł jako wskaźnik.

			\spacing

\begin{lstlisting}[label={lst:ast-node}]
struct ast_node {
	enum ast_type  type;    /* Rozrozniamy typ wedlug tej flagi */
	void          *ast;     /* ast_num, ast_for, ast_while, et cetera */
	uint16_t       line_no;
	uint16_t       col_no;
};
\end{lstlisting}
			
			\spacing
			
			Konkretne węzły definiujemy w następujący sposób:

			\spacing

\begin{lstlisting}[label={lst:ast-concrete-node}]
struct ast_num {
	int32_t value;
};
\end{lstlisting}
			
			\spacing
			
			Taki AST stanowi strukturę drzewiastą, mającą wszystkie zalety i wady drzew jako struktur
			danych. Mając takie drzewo, jesteśmy w stanie prowadzić zwykłe przeszukiwanie w głąb
			i wszerz. W danym przypadku taki algorytm się nazywa \textbf{AST visitor}. Dokładnie w ten
			sposób działa każda z przedstawionych niżej analiz semantycznych
			oraz generacja kodu pośredniego.

			\spacing

			\begin{algorithm}
				\caption{Przeszukiwanie AST}
				\begin{algorithmic}[1]

				\Procedure{DFS}{AST}
					\For{each child node Child of AST}
						\State DFS(Child)
					\EndFor
				\EndProcedure

				\end{algorithmic}
			\end{algorithm}

		\subsection{Implementacja analizatora składniowego}
			
			W danym przypadku, analizator składniowy jest napisany ręcznie, chociaż są narzędzia od
			projektu GNU, takie jak GNU Bison i UNIX'owe, takie jak YACC. Niniejszy analizator
		  	jest napisany bez pomocy tych programów, aby jawnie pokazać, jak się przekładają
		  	produkcje BNF na język C.
		  	\\
		  	
		  	Aby poradzić sobie z zadaniem pisania takiego analizatora, możemy zauważyć, że
		  	zadanie to sprowadza się do implementacji każdej produkcji gramatycznej osobno.
	  	
	  	\subsection{Reprezentacja wizualna AST}
	  		
	  		Jest pokazana też implementacja \textbf{visitor}'u, pozwalającego na przeprowadzenie AST
			do formy tekstowej. Do tego służy funkcja \textbf{ast_dump()}. Przyjmuje ona wskaźnik do
			węzła drewa i działa według algorytmu DFS, opisanego wyżej, przy tym pisząc tekstową formę
			węzłów do pliku (ewentualnie, do \texttt{stdout}). Funkcjonalność ta jest bardzo ważna do
			prowadzenia testów jednostkowych samego AST oraz analizatoru składniowego.
			Niżej pokazany jest przykładowy wynik działania tej funkcji.
			\\
	
			\lstinputlisting[label={lst:ast-example}]{code/ast\_example.txt}

		\subsection{Poniżanie poziomu abstrakcji}
			
			Niniejszy język zapewnia mechanizm do konwertacji konkretnych
			węzłów AST na inne, mające więcej szczegółów, prostszych dla
			dalszej analizy. Zadanie to polega na przejściu AST i zrobieniu
			podmiany pętli \texttt{for} jednego rodzaju na drugą. Semantyka i
			ciało pętli pozostaje bez zmian.
			\\

			Wzięto pod uwagę tworzenie zagnieżdżonych pętli
			takiego rodzaju. Wtedy ukryty iterator pętli dostaje kolejny numer
			w nazwie zmiennej. Dla pętli o głębokości 1 \texttt{__i1}, dla kolejnych
			\texttt{__i2, __i3} i t.d..
			\\
			
			Jako przykład podana pętla for \texttt{ForRangeStmt}. Dana konstrukcja i jej AST

			\lstinputlisting[label={lst:ast-code-before-lower}]{code/ast\_src\_before\_lower.txt}
			\lstinputlisting[label={lst:ast-before-lower}]{code/ast\_before\_lower.txt}

			... przyjmie następującą postać

			\lstinputlisting[label={lst:ast-code-after-lower}]{code/ast\_src\_after\_lower.txt}
			\lstinputlisting[label={lst:ast-after-lower}]{code/ast\_after\_lower.txt}	        
			\newpage

	        Zauważmy, że często kompilatory podczas zmiany poszczególnych elementów AST
	        precyzują semantykę poprzez dodanie "ukrytych" węzłów drzewa. Tak Clang
	        dodaje szczegóły odnośnie konwersji typów. W podanym przykładzie to
	        \texttt{InitListExpr} i \texttt{ImplicitCastExpr}. Wtedy ten proces raczej
	        się nazywa dodaniem semantyki niż poniżaniem poziomu abstrakcji.

			\lstinputlisting[label={lst:ast-src-clang}]{code/ast\_src\_clang.txt}
			\lstinputlisting[label={lst:ast-clang}]{code/ast\_clang.txt}
