\documentclass[a4paper, 12pt]{extarticle}

\usepackage[utf8]{inputenc}
\usepackage[polish]{babel}
\usepackage{polski}
\usepackage[fleqn]{amsmath}  % Aligned to left equations.
\usepackage[a4paper, total={7.5in, 9.5in}]{geometry}
\usepackage{indentfirst}
\usepackage{amsfonts}

\usepackage{caption}
\usepackage{chngcntr}
\usepackage{cleveref}

\usepackage{geometry}

\usepackage{syntax}
\usepackage{fancyhdr}
\usepackage{mathpartir}

\usepackage{array}
\usepackage{listings}

\usepackage{algorithm}
\usepackage{algpseudocode}

\usepackage{tikz}

\usetikzlibrary{automata, positioning, arrows}

\algnewcommand\algorithmicswitch{\textbf{switch}}
\algnewcommand\algorithmiccase{\textbf{case}}
\algnewcommand\algorithmicassert{\texttt{assert}}
\algnewcommand\Assert[1]{\State \algorithmicassert(#1)}%
% New "environments"
\algdef{SE}[SWITCH]{Switch}{EndSwitch}[1]{\algorithmicswitch\ #1\ \algorithmicdo}{\algorithmicend\ \algorithmicswitch}%
\algdef{SE}[CASE]{Case}{EndCase}[1]{\algorithmiccase\ #1}{\algorithmicend\ \algorithmiccase}%
\algtext*{EndSwitch}%
\algtext*{EndCase}%

\lstset{
    xleftmargin    = .2\textwidth,
    xrightmargin   = .2\textwidth,
    basicstyle     = \footnotesize\ttfamily,
    breaklines     = true,
    linewidth      = 16cm
}

\newcolumntype{L}[1]{>{\raggedright\let\newline\\\arraybackslash\hspace{0pt}}m{#1}}
\newcolumntype{C}[1]{>{\centering  \let\newline\\\arraybackslash\hspace{0pt}}m{#1}}
\newcolumntype{R}[1]{>{\raggedleft \let\newline\\\arraybackslash\hspace{0pt}}m{#1}}

\newcommand{\spacing}{\vskip 0.5cm}

\let\titleoriginal\title           % save original \title macro
\renewcommand{\title}[1]{          % substitute for a new \title
    \titleoriginal{#1}%            % define the real title
    \newcommand{\thetitle}{#1}     % define \thetitle
}

\lstset{
	firstnumber = 0,
	tabsize     = 4,
	frame       = single,
	linewidth   = 1.2\textwidth,
	xleftmargin = .5cm
}

\captionsetup{
    format=plain,
    singlelinecheck=false,
    justification=raggedright,
    margin={.5cm, 0cm}
}

\counterwithin{figure}{section}
\counterwithin{table}{section}

\title{Kompilator języka strukturalnego}
\author{David Korenchuk}
\date{August, 23, 2023}

\begin{document}
    \pagestyle{fancy}
    \lhead{\leftmark}
    \rhead{\thetitle}

    \maketitle
    
    \newpage
    
    \tableofcontents

    \newpage
	
	%% 1.   Wprowadzenie
	%% 2.   Teoria
	%% ...  Opis lex
	%% ...  Opis parse
	%% ...  Opis codegen
	%% ...  Opis interpreter
	%% N.   Podsumowanie. Co było zrobione
	
	\section{Wprowadzenie}
	
		W dniu dzisiejszym istnieje wiele języków programowania. Przyczyna na istnienie narzędzia takiego
		rodzaju jest taka, że człowiek myśli i mówi zdaniami. Aby móc przeprowadzić ludzkie zdania do
		formy, zrozumiałej do komputera, niezbędnie te zdania muszą być przeprowadzone do zestawu dużo
		prostszych zdań, niż w ludzkich językach.
		\\

		...

		%% - Historia idei kompilatorów, języków programowania
		%% - Pierwsze z nich
		%% - Sposoby, przez które były zaimplementowane
		%% - Powstanie nowych \ użycie starych narzędzi matematycznych
		%% - Dzisiejsza teoria języków regularnych

		\newpage
		
	\section{Teoria}

		\subsection{Języki formalne}

			Według teorii automatów, automat -- jest to jednostka wykonawcza. Jednostki te, zależnie od swojej
			struktury i tego, jaki \textbf{język formalny} oni mogą obrobić, dzielą się na klasy.
			\\
			
			Klasy te opisane są \textbf{hierarchią Chomsky’ego}. Mówi ona o tym, że języki formalne dzielą się na
			4 typy:
			\begin{itemize}
				\item Typ 3 -- języki regularne
				\item Typ 2 -- języki bezkontekstowe
				\item Typ 1 -- języki kontekstowe
				\item Typ 0 -- języki rekurencyjnie przeliczalne
			\end{itemize}
			
			\spacing

			Jako przykład języka typu 3 według hierarchii Chomsky'ego można podać wyrażenia regularne. Język ten
			opisuje się automatem skończonym deterministycznym (DFA). Bardziej szczegółowo wyrażenia regularne będą
			rozpatrzone w opisaniu analizy leksykalnej.
			
		\subsection{Klasyfikacja gramatyczna}

			Niniejszy język nie może być odniesiony do żadnej z klas hierarchii Chomsky'ego, chociaż jest on
			językiem regularnym. Tak jest dlatego, że można napisać gramatycznie poprawny kod, który jednak prowadzi
			do błędów kontekstowych i logicznych. Naprzykład 
			
			\spacing
			
			\begin{lstlisting}[caption={}, label={lst:ambigous-production}]
	void f() {
		 return argument + 1;
	}
			\end{lstlisting}

			\spacing
			
			Kolejną z przyczyn niemożliwości odniesienia naszego języka do jednej z klas hierarchii Chomsky'ego
			jest niejednoznaczność konstrukcji językowych. Przykład niżej pokazuje, że nie można jednoznacznie
			stwierdzić, czy \texttt{data * d} jest deklaracją zmiennej albo operatorem mnożenia dwóch zmiennych.
			Aby móc poprawnie prowadzić analizę syntaksyczną, musimy zadbać o rozróżnienie kontekstu.

			\spacing
			
			\begin{lstlisting}[caption={}, label={lst:ambigous-production}]
	void f() {
		 data *d;
	}
			\end{lstlisting}

			\newpage

	\section{Analiza leksykalna}

		Jednym ze sposobów na sprowadzanie kodu źródłowego do postaći listy tokenów jest narzędzie flex.
		Przyjmuje ono zestaw reguł w postaći wyrażeń regularnych, według których
		działa rozbicie tekstu wejściowego. Można jednak ominąć lex i zaimplemenetować lexer ręcznie, ale
		ta praca nie skupia się na tym.

		\subsection{Wyrażenia regularne}

			Wyrażenie regularne -- łańcuch znaków, zawierający pewne polecenia do wyszukiwania tekstu.
			\\

			Mówimy, że wyrażenie regularne określone nad alfabetem $\Sigma$, jeżeli zachodzą następujące
			warunki:

			\begin{itemize}
				\item $\emptyset$ -- wyrażenie regularne, reprezentujące pusty zbiór.
				\item $\epsilon$ -- wyrażenie regularne, reprezentujące pusty łańcuch.
				\item $\forall_{a \in \Sigma}$, $a$ reprezentuje jeden znak.
				\item Warunek indukcyjny: jeżeli $R_1, R_2$ -- wyrażenia regularne, $(R_1 R_2)$
				      stanowi konkatenację $R_1$ i $R_2$.
		      	\item Warunek indukcyjny: jeżeli $R$ -- wyrażenie regularne, $R*$ stanowi domknięcie Kleene'ego. 
			\end{itemize}

			\spacing

			W rzeczywistości, takich zasad może być więcej.
			\\

			Zazwyczaj wyrażenie regularne jest realizowane za pomocą DFA (Deterministic finite automaton,
			Deterministyczny automat skończony).
			Lex sprowadza podany zbiór zasad do takiego automatu.
			\\

			Podamy przykład automatu dla wyrażenia \texttt{-?[0-9]+}
			
			\spacing
			\spacing

			\begin{center}
			\begin{tikzpicture}[node distance=4cm]
				\node[state, initial, accepting] (0) {0};
				\node[state, right of=0]         (1) {1};
				\node[state, right of=1]         (2) {2};
				\draw
				(0) edge[       bend left ]             (1)
				(0) edge[below, bend right] node{$-$}   (1)
				(1) edge[below            ] node{$0-9$} (2)
				(2) edge[loop below       ] node{$0-9$} (2)
				;
			\end{tikzpicture}
			\end{center}
			
			Aby odśledzić wykonane kroki, można wypełnić tabelę przejść pomiędzy stanami. Podamy przykład dla
			łańcucha \texttt{-22}

			\spacing
			\spacing

		\begin{center}
			\setlength{\tabcolsep}{0.5em}
			\renewcommand{\arraystretch}{1.2}
			\begin{tabular}{ | L{3cm} | L{4cm} | }
				\hline
				Biężący stan        & Akcja \\
				\hline
				\textbf{0}          & zaakceptować - \\
				\hline
				0, \textbf{1}       & zaakceptować 2 \\
				\hline
				0, 1, \textbf{2}    & zaakceptować 2 \\
				\hline
			\end{tabular}
		\end{center}
		
		\newpage

	\subsection{Flex}

		Flex jest narzędziem projektu GNU. Pozwala ono w wygodny sposób podać listę reguł dla analizy
		leksykalnej (ang. Scanning). Flex jest mocno powiązany z językiem C, dlatego program w flex'u
		korzysta z konstrukcji języka C. Pokażemy przykład użycia flex'u
		
		\spacing
		
		\begin{lstlisting}[caption={Przykład użycia flex}, label={lst:flex-example}]
%{
#include "portrzebny-do-analizy-plik.h"

/* Kod w jezyku C. */
%}

/* Opcje flex */
%option noyywrap nounput noinput
%option yylineno

%% /* Reguly w postaci wyrazen regularnych. */

/*********************************************************************/
/* Wzorzec                   | Akcja przy znalezieniu takiego wzorcu */
/*********************************************************************/
-?[0-9]+                       LEX_CONSUME_WORD(TOK_INTEGRAL_LITERAL)
-?[0-9]+\.[0-9]+               LEX_CONSUME_WORD(TOK_FLOATING_POINT_LITERAL)
\"([^\"\\]*(\\.[^\"\\]*)*)\"   LEX_CONSUME_WORD(TOK_STRING_LITERAL)
\'.\'                          LEX_CONSUME_WORD(TOK_CHAR_LITERAL)

.                              { /* Znaleziony niewiadomy znak.
                                    Zglosic blad.
                                  */ }
%%
		\end{lstlisting}

		\spacing
		
		Zauważmy, że flex próbuje szukać wzorców w tekscie dokładnie w takiej kolejności, która jest podana w
		jego kodzie. Dlatego często robią ostatnią regułe z wyrażeniem regularnym ".", który akceptuje
		dowolny znak, i umieszczają tam komunikat o błędzie.
		\\

		W naszym przypadku, lex generuje kod, który gromadzi wszystkie znalezione lexemy do tablicy.

		\newpage

	\section{Analiza syntaksyczna}
		
		\subsection{Definicja}

			Mając listę składników elementarnych wejściowego programu, jesteśmy w stanie przejść do
			następnego etapu kompilacji -- analizy syntaksycznej. Jest to proces generacji struktury
			drzewiastej, a mianowicie AST (Abstract Syntax Tree).
			\\

			AST może być stworzony po zdefiniowaniu gramatyki regularnej danego języka. Stosuje się
			do tego notacja BNF (Backus–Naur form). Pełny opis gramatyki pokazany jest w końcu pracy. Pokażemy
			tylko kilka przykładów:
			
			\spacing

		    \setlength{\grammarindent}{12em}

		    \begin{grammar}
		        <program> ::= ( <function-decl> | <structure-decl> )*

		        <var-decl> ::= <type> ( \tt{*} )* <id> \tt{=} <logical-or-stmt> \tt{;}

		        <stmt> ::= <block-stmt>
		        \alt <selection-stmt>
		        \alt <iteration-stmt>
		        \alt <jump-stmt>
		        \alt <decl>
		        \alt <expr>
		        \alt <assignment-stmt>
		        \alt <primary-stmt>
			\end{grammar}
			
			\spacing
			
		\subsection{Znane problemy}

			Projektując gramatykę, należy wziąć pod uwagę problem rekurencji lewej (Left recursion).
			Są produkcje gramatyczne, nie pozwalające kodu, które je implementuje przejść do następnego
			terminalu, stosując tą samą produkcję, co prowadzi do rekurencji nieskończonej.
			\\
			
			Rekuręcja lewa może wyglądać następująco:

		    \begin{grammar}
		    	<factor> ::= <factor> '+' <term>
			\end{grammar}

			\spacing

			Kod, wykonujący tą regułe będzie miał postać:
			
			\begin{lstlisting}[caption={Rekurencja lewa}, label={lst:left-recursion}]
void factor() {
	factor(); // Rekurencja bez zadnego warunku wyjscia
	consume('+');
	term();
}
			\end{lstlisting}

			\newpage
	
	\section{Analiza semantyczna}

		Aby zapewnić poprawność napisanego kodu, stosuje się wiele rodzajów analiz. Niniejszy kompilator
		dysponuje trzema:
		
		\begin{itemize}
			\item Analiza nieużytych zmiennych, oraz zmiennych, które są zdefiniowane, ale nie zostały użyte
			\item Analiza poprawności typów
			\item Analiza prawidłowego użycia funkcji
		\end{itemize}
		
		\subsection{Analiza nieużywanych zmiennych}
			
			Podamy przykłady kodu prowadzącego do odpowiednich ostrzeżeń
			
			\spacing

			\begin{lstlisting}[caption={Nieużywana zmienna}, label={lst:warn-unused-var}]
void f() {
	int argument = 0; // Warning: unused variable `argument`
}
			\end{lstlisting}

			\begin{lstlisting}[caption={Nieużywany parametr}, label={lst:warn-unused-var}]
void f(int argument) {} // Warning: unused variable `argument`
			\end{lstlisting}
			
			\begin{lstlisting}[caption={Nieodczytana zmienna}, label={lst:warn-unused-var}]
void f() {
	int argument = 0;
	++argument; // Warning: variable `argument` written, but never read
}
			\end{lstlisting}

			%% (\ref{lst:warn-unused-var})
			
			\spacing
			
			Rzecz polega na przejściu drzewa syntaksycznego i zwiększania liczników
			\textbf{read_uses} i \textbf{write_uses} dla każdego węzła typu
			\textbf{ast_sym}.
			\\
			
			Algorytm operuje na blokach kodu, zawartego w \textbf{\{ ... \}}. Po przejściu każdego bloku
			(w tym rekurencyjnie), analiza jest wykonana w następujący sposób:

			\begin{algorithm}
				\caption{Wyszukiwanie nieużywanych zmiennych}
				\begin{algorithmic}[1]

				\Procedure{Analyze}{AST}
					\State Set $\gets$ all declarations at current scope depth

					\For{each collected declaration Use in Set}
						\If{Use is not a function \& Use.ReadUses is 0}
							\State Emit warning
						\EndIf
					\EndFor
				\EndProcedure

				\end{algorithmic}
			\end{algorithm}
			
			Do analizy nieużywanych funkcji stosuje się tem sam algorytm. Jedyne, co jest wtedy zmienione --
			sprawdzenie, czy nazwa rozpatrywanej funkcji nie jest \textbf{main}. Funkcja \textbf{main} jest
			wywołana automatycznie.
		
		\subsection{Analiza typów}
		
			Podczas analizy typów należy sprawdzić, czy:

			\begin{itemize}
				\item Oba operandy wyrażenia binarnego mają ten sam typ
				\item Faktycznie zwracana z funkcji wartość zgadza się z jej
				      deklaracją
		      	\item Prawidłowa ilość i typy argumentów byli przekazane do wyrażenia
		      	      wywołania funkcji
      	      	\item Rozmiar zadeklarowanej tablicy jest dopuszczalny
      	      	\item Indeks tablicy nie wykracza poza jej rozmiar (w przypadku stałego indeksu)
			\end{itemize}
		
		\spacing

		Analiza typów w naszym przypadku jest dość prosta i tak samo polega na przejściu drzewa
		syntaksycznego.
		
		\subsubsection{Matematyczny opis systemu typów}

		\begin{mathpar}
		\inferrule
		  {\Gamma \vdash \diamond}
		  {\Gamma \vdash \text{true} : bool}
		\quad

		\inferrule
		  {\Gamma \vdash \diamond}
		  {\Gamma \vdash \text{false} : bool}
		\end{mathpar}

		\begin{mathpar}
		\inferrule
		  {\Gamma \vdash \diamond}
		  {\Gamma \vdash n : int}
		\end{mathpar}

		\begin{mathpar}
		\inferrule
		  {\Gamma \vdash \diamond}
		  {\Gamma \vdash c : char}
		\end{mathpar}

		\begin{mathpar}
		\inferrule
		  {\Gamma \vdash \diamond}
		  {\Gamma \vdash x : float}
		\end{mathpar}

		\spacing

Oznaczmy tutaj dla $\mathbb{N, R}: \oplus \in \{ +, -, *, /, <, >, \le, \ge, ==, \neq, ||, \&\& \}$, wtedy

		\spacing

		\begin{mathpar}
		\inferrule
		  {\Gamma \vdash e_1 : float \\ \Gamma \vdash e_2 : float}
		  {\Gamma \vdash e_1 \oplus e_2 : float}
		\end{mathpar}

		\spacing

Dodamy do $\oplus$ operacje tylko dla $\mathbb{N}: \oplus \ \cup \ \{ |, \&, \string^, <<, >>, \% \}$, wtedy

		\spacing

		\begin{mathpar}
		\inferrule
		  {\Gamma \vdash e_1 : int \\ \Gamma \vdash e_2 : int}
		  {\Gamma \vdash e_1 \oplus e_2 : int}
		\end{mathpar}

		\begin{mathpar}
		\inferrule
		  {\Gamma \vdash \text{condition} : bool \\ \Gamma \vdash e_1 : \tau \\ \Gamma \vdash e_2 : \tau}
		  {\Gamma \vdash \text{if (condition) } \{ \ e_1 \ \}  \text{ else }  \{ \ e_2 \ \} : \tau}
		\end{mathpar}

		\begin{mathpar}
		\inferrule
			{\Gamma \vdash \text{condition} : bool \\ \Gamma \vdash e : \tau}
			{\Gamma \vdash \text{while (condition) } \{ \ e \ \} : \tau}
		\end{mathpar}

		\begin{mathpar}
		\inferrule
			{\Gamma \vdash \text{condition} : bool \\ \Gamma \vdash e : \tau}
			{\Gamma \vdash \text{do } \{ \ e \ \} \text{ while (condition)} : \tau}
		\end{mathpar}

		\begin{mathpar}
		\inferrule
		  {\Gamma(x) = \tau}
		  {\Gamma \vdash x : \tau}
		\end{mathpar}

		\newpage


	\section{Generacja warstwy pośredniej}
		
		
		
		\newpage

	\section{Interpreter}
		
		
		
		\newpage

	\begin{thebibliography}{9}
		\bibitem{texbook}
			https://www.bates.edu/biology/files/2010/06/How-to-Write-Guide-v10-2014.pdf
		
		\bibitem{texbook}
			http://lucacardelli.name/papers/typesystems.pdf
	\end{thebibliography}

\end{document}